%%%%%%%%%%%%%%%%%%%%%%%%%%%%%%%%%%%%%%%%%%%%%%%%%%%%%%%%%%%%%%%%%%%%%%%%%%%%%%%%%%%%%%%%%%%%%%%
%Plantilla: para la realizaci�n de informes.
%Curso:     Simulaci�n estad�stica.
%Profesor:  Johann A. Ospina.
%%%%%%%%%%%%%%%%%%%%%%%%%%%%%%%%%%%%%%%%%%%%%%%%%%%%%%%%%%%%%%%%%%%%%%%%%%%%%%%%%%%%%%%%%%%%%%%


%Establece el tipo de documento (art�culo), tama�o de letra (10pt) a una columna.
\documentclass[letterpaper,12pt,onecolumn,titlepage]{article} 
 
 
% Cargar paquetes
\usepackage{verbatim}
\usepackage{mathrsfs}
\usepackage{amsmath}
\usepackage{amssymb}
\usepackage{subfigure}
\usepackage{ucs}
\usepackage[latin1]{inputenc}
\usepackage[spanish]{babel}
\usepackage{fontenc}
\usepackage{graphicx}
\usepackage{anysize}
\usepackage{fancyhdr}
\usepackage[comma,authoryear]{natbib}
\usepackage{url} %paquete para definir url
\usepackage{hyperref}  %hipervinculos

%Estilo de la p�gina
\pagestyle{fancy}

%Establecer el margen
\marginsize{2cm}{2cm}{1cm}{1cm}
\setlength{\headheight}{13.1pt}


% Portada
\title{
    \textbf{Laboratorio N.2}\
    ~\\{Introduccion a Los Metodos Estadisticos}   
    ~\\{Generacion de Estimadores}}
\author{
    {Diana Carolina Arias Sinisterra Cod. 1528008}
 ~\\{Kevin Steven Garcia Chica Cod. 1533173}
 ~\\{Cesar Andres Saavedra Vanegas Cod. 1628466}}

\date{
     \textbf{Universidad Del Valle}\   
    ~\\{Facultad De Ingenieria}
    ~\\{Estadistica}
    ~\\{Octubre}
    ~\\{2017}}
 
 
 
\decimalpoint %Poner punto decimal
 
\begin{document}
 
% Se aplica el formato a las p�ginas. Se despliegan: portada e �ndices de materias, figuras y tablas
\renewcommand{\listtablename}{}
\renewcommand{\tablename}{Tabla}
\maketitle
\setcounter{page}{2}
\tableofcontents{}
%\thispagestyle{empty}
%\newpage
\listoffigures{}
\listoftables

\thispagestyle{empty}

\newpage
\fancyhead{}
\fancyfoot{}
 
% Encabezado y pie de pagina
\lhead{Introduccion a los Metodos Estadisticos}
\lfoot{Universidad Del Valle}
\rfoot{\thepage}

% Estilo de la bibliograf�a
\bibliographystyle{apalike}
 
% Desarrollo de los contenidos del documento
\pagebreak\section{Situaci\'{o}n 1}
\subsection{Punto a.}
\subsection{Punto b.}
\subsection{Punto c.}

\pagebreak\section{Situaci\'{o}n 2}
\subsection{Punto a.}
\subsection{Punto b.} 
\subsection{Punto c.} 


\pagebreak\section{Situaci\'{o}n 3}
\subsection{Punto a.}


\pagebreak\section{Situaci\'{o}n 4}
\subsection{Punto a.}
\subsection{Punto b.}
\subsection{Punto c.}

\pagebreak\section{Situaci\'{o}n 5}
\subsection{Punto a.}

\pagebreak\section{Situaci\'{o}n 6}
\subsection{Punto a.}

\pagebreak\section{Situaci\'{o}n 7}
~\ Sean $Y_{1}, Y_{2}, Y_{3},...,Y_{n}$ una muestra aleatoria extraida de una poblacion con funcion de densidad:

~\ $$f({x})= \frac{1}{2\theta +2}  ;  -1<Y<2\theta+1$$
\subsection{Punto a.}
~\\ Un estimador maximo verosimil para $\theta$ y $\sigma^{2}$ son:

~\\ \textbf{Para $\theta$ :}

~\\ $L(y;\theta) = \prod_{i=1}^n{}(\frac{1}{2\theta +2})$
~\\ $L(y;\theta) = (\frac{1}{2\theta +2})^{n}$
~\\ $Ln(L(y;\theta)) = Ln((\frac{1}{2\theta +2})^{n})$
~\\ $L(y;\theta) = n[Ln((\frac{1}{2\theta +2})]$
~\\ $L(y;\theta) = n[Ln(1)-Ln({2\theta+2})]$
~\\ $L(y;\theta) = n[-Ln({2\theta+2})]$
~\\ $\frac{dL(y;\theta)}{\theta} = \frac{d}{\theta}(n[-Ln({2\theta+2}))]$

~\\ $\hat{\theta} = \frac{n}{\theta+1}$

~\\ Donde el parametro es el limite superior de la variacion de la funcion de distribucion. 

~\\ $\therefore \hat{\theta} = Maximo = Y = [Y_{1}, Y_{2}, Y_{3},...,Y_{n}]$ 

~\\ \textbf{Para $\sigma^{2}$ :}

~\\ 
~\\
~\\
~\\


\subsection{Punto b.}

\bibliography{Bibliografia}
\end{document}