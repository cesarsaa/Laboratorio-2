%%%%%%%%%%%%%%%%%%%%%%%%%%%%%%%%%%%%%%%%%%%%%%%%%%%%%%%%%%%%%%%%%%%%%%%%%%%%%%%%%%%%%%%%%%%%%%%
%Plantilla: para la realizaci�n de informes.
%Curso:     Simulaci�n estad�stica.
%Profesor:  Johann A. Ospina.
%%%%%%%%%%%%%%%%%%%%%%%%%%%%%%%%%%%%%%%%%%%%%%%%%%%%%%%%%%%%%%%%%%%%%%%%%%%%%%%%%%%%%%%%%%%%%%%


%Establece el tipo de documento (art�culo), tama�o de letra (10pt) a una columna.
\documentclass[letterpaper,12pt,onecolumn,titlepage]{article} 
 
 
% Cargar paquetes
\usepackage{verbatim}
\usepackage{mathrsfs}
\usepackage{amsmath}
\usepackage{amssymb}
\usepackage{subfigure}
\usepackage{ucs}
\usepackage[latin1]{inputenc}
\usepackage[spanish]{babel}
\usepackage{fontenc}
\usepackage{graphicx}
\usepackage{anysize}
\usepackage{fancyhdr}
\usepackage[comma,authoryear]{natbib}
\usepackage{url} %paquete para definir url
\usepackage{hyperref}  %hipervinculos

%Estilo de la p�gina
\pagestyle{fancy}

%Establecer el margen
\marginsize{2cm}{2cm}{1cm}{1cm}
\setlength{\headheight}{13.1pt}


% Portada
\title{
    \textbf{Laboratorio N.2}\
    ~\\{Introduccion a Los Metodos Estadisticos}   
    ~\\{Generacion de Estimadores}}
\author{
    {Diana Carolina Arias Sinisterra Cod. 1528008}
 ~\\{Kevin Steven Garcia Chica Cod. 1533173}
 ~\\{Cesar Andres Saavedra Vanegas Cod. 1628466}}

\date{
     \textbf{Universidad Del Valle}\   
    ~\\{Facultad De Ingenieria}
    ~\\{Estadistica}
    ~\\{Octubre}
    ~\\{2017}}
 
 
 
\decimalpoint %Poner punto decimal
 
\begin{document}
 
% Se aplica el formato a las p�ginas. Se despliegan: portada e �ndices de materias, figuras y tablas
\renewcommand{\listtablename}{}
\renewcommand{\tablename}{Tabla}
\maketitle
\setcounter{page}{2}
\tableofcontents{}
%\thispagestyle{empty}
%\newpage
%\listoffigures{}
%\listoftables

\thispagestyle{empty}

\newpage
\fancyhead{}
\fancyfoot{}
 
% Encabezado y pie de pagina
\lhead{Introduccion a los Metodos Estadisticos}
\lfoot{Universidad Del Valle}
\rfoot{\thepage}

% Estilo de la bibliograf�a
\bibliographystyle{apalike}
 
% Desarrollo de los contenidos del documento
\pagebreak\section{Situaci\'{o}n 1}
\subsection{Punto a.}
~\\ Un estimador maximo verosimil de $\lambda$ para una funcion Poisson$(\lambda)$ esta dado por.

~\\ $$f_{(x)}(x) = \frac{\exp^{-\lambda}\lambda^{X}}{X!}$$

~\\ $L(x,\lambda) = \prod_{i=1}^n{}\frac{\exp^{\lambda}\lambda^{X}}{X!}$

~\\ $L(x,\lambda) = \frac{\exp^{\lambda n}\lambda^{\sum{X}}}{X!}$
~\\ $Ln(L(x,\lambda)) = Ln(\frac{\exp^{\lambda n}\lambda^{\sum{X}}}{X!})$

~\\ $L(x,\lambda) = (-\lambda n) + Ln(\lambda{\sum{x_{i}}}) - (Ln\sum{x_{i}})$
~\\ $L(x,\lambda) = -(\lambda n) + {\sum{x_{i}}}Ln(\lambda) - (Ln\sum{x_{i}})$

~\\ $\frac{dL(x;\lambda)}{d\lambda} = \frac{d}{d\lambda}(-\lambda n + {\sum{x_{i}}}Ln(\lambda) - (Ln\sum{x_{i}}))$

~\\ $L(x,\lambda) = \frac{\sum{x_{i}}}{\lambda} - n $
~\\ $\frac{\sum{x_{i}}}{\lambda} - n = 0$

~\\ $\frac{\sum{x_{i}}}{\lambda} = n $
~\\ $\hat{\lambda} = \frac{\sum{x_{i}}}{n}$

~\\ \textbf{Donde $\hat{\lambda}$ es un estimador maximo verosimil e insesgado para la funcion de distribucion poisson.} 
 
~\\ $\therefore \hat{\lambda} = \bar{x}$

\subsection{Punto b.}
~\\ En un estimador insesgado puesto que la esperanza es igual al parametro;

~\\ $E[\hat{\lambda}] = E[\frac{\sum{x_{i}}}{n}]$
~\\ $E[\hat{\lambda}] = \frac{1}{n}E[\sum{x_{i}}]$
~\\ $E[\hat{\lambda}] = \frac{1}{n}(\sum)E[x]$

~\\ $E[\hat{\lambda}] = E[x]$
~\\ $\hat{\lambda} = \bar{x}$

~\\ Donde $\hat{\lambda}$ es un estimador insesgado para la funncion poisson de parametro $(\lambda)$. 

~\\ La varianza esta dada por: 

~\\ $Var[\hat{\lambda}]= var[\frac{\sum{x_{i}}}{n}]$
~\\ $Var[\hat{\lambda}]= \frac{1}{n^2} var[\sum{x_{i}}]$
~\\ $Var[\hat{\lambda}]= \frac{1}{n} var[x_{i}]$

~\\ $Var[x_{i}]= \frac{\lambda}{n}$

\subsection{Punto c.}
~\\ Para clacular la probabilidad de que en un dia particular se reciban maximo 2 quejas, es decir $P[x<2|\hat{y}=3]$ a partir de la muestra que que cuenta con una media de $\hat{y}=3$ se usa la funcion de densidad de la distribucion de poisson con parametro $\lambda=3$. 

~\\ $P[x\le2]= \frac{\exp^{-\lambda}\lambda^{X}}{X!}$

~\\ $P[x\le2]= \frac{\exp^{-3}3^{0}}{0!} + 
			 \frac{\exp^{-3}3^{1}}{1!} + 
			 \frac{\exp^{-3}3^{2}}{2!}$
			 
~\\ $P[x\le2]= 0.4231 $

~\\ Por lo cual la probabilidad que la tiene oficina de recibir como maximo dos quejas en un dia es del {$42.31\%$}

\pagebreak\section{Situaci\'{o}n 2}
\subsection{Punto a.}
\subsection{Punto b.} 
\subsection{Punto c.} 

\pagebreak\section{Situaci\'{o}n 4}
\subsection{Punto a.}
\subsection{Punto b.}
\subsection{Punto c.}

\pagebreak\section{Situaci\'{o}n 5}
\subsection{Punto a.}
~\\ $$f(x;\theta) = \frac{2\theta^2}{x^3} ; \theta<x<\infty $$
~\\ $M1'=\sum_{i=1}^{n}\frac{x_{i}}{n}=\bar{x}$
~\\ $\mu'_1 =?$
~\\ $\mu'_1=E[X]= \int \limits_{\theta}^{\infty} x f(x) \cdot dx$
~\\ $E[X]=\int \limits_{\theta}^{\infty} x \frac{2\theta^2}{x^3}\cdot dx=\int \limits_{\theta}^{\infty}\frac{2\theta^2}{x^2}\cdot dx$
~\\ $E[X]=2\theta^2 \int \limits_{\theta}^{\infty}\frac{1}{x^2}\cdot dx= 2\theta^2[-\frac{1}{x}|{{^\infty}{_\theta}}]=2\theta^2(\frac{1}{\theta})=2\theta$
~\\ $$\mu'_1=E[X]=2\theta=\bar{X}=M1'$$
~\ $$\hat{\theta}=\frac{\bar{X}}{2}$$
~\\ \textbf{En conclusion, el estimador por el metodo de los momentos para $\theta$ de la funcion de densidad $f(x;\theta) = \frac{2\theta^2}{x^3} ; \theta<x<\infty $ es $\hat{\theta}=\frac{\bar{X}}{2}$}


\pagebreak\section{Situaci\'{o}n 7}
~\ Sean $Y_{1}, Y_{2}, Y_{3},...,Y_{n}$ una muestra aleatoria extraida de una poblacion con funcion de densidad:

~\ $$f({y})= \frac{1}{2\theta +2}  ;  -1<Y<2\theta+1$$
~\ Donde; $f({y}) ~ Uniforme(a=-1,b=2\theta+1)$

\subsection{Punto a.}
~\ Un estimador maximo verosimil para $\theta$ y $\sigma^{2}$ son:

~\\ \textbf{Para $\theta$ :}

~\\ $L(y;\theta) = \prod_{i=1}^n{}(\frac{1}{2\theta +2})$
~\\ $L(y;\theta) = (\frac{1}{2\theta +2})^{n}$
~\\ $Ln(L(y;\theta)) = Ln((\frac{1}{2\theta +2})^{n})$
~\\ $L(y;\theta) = n[Ln((\frac{1}{2\theta +2})]$
~\\ $L(y;\theta) = n[Ln(1)-Ln({2\theta+2})]$
~\\ $L(y;\theta) = n[-Ln({2\theta+2})]$
~\\ $\frac{dL(y;\theta)}{\theta} = \frac{d}{\theta}(n[-Ln({2\theta+2}))]$

~\\ $\hat{\theta} = \frac{n}{\theta+1}$

~\\ Donde el parametro es el limite superior de la variacion de la funcion de distribucion. 

~\\ $\therefore \hat{\theta} = Maximo = [Y_{1}, Y_{2}, Y_{3},...,Y_{n}]$ 

~\\ \textbf{Para $\sigma^{2}$ :}

~\\ $\sigma^{2} = Var(Y) = \frac{(b-a)^2}{12}$
~\\ $Var(Y) = \frac{(2\theta+1-(-1))^2}{12}$
~\\ $Var(Y) = \frac{(2\theta+2))^2}{12}$
~\\ $L(y;\sigma^{2}) = \prod_{i=1}^n{} \frac{(2\theta+2))^2}{12}$
~\\ $L(y;\sigma^{2}) = (\frac{(2\theta+2))^2}{12})^{n}$
~\\ $Ln(L(y;\sigma^{2})) = Ln(\frac{(2\theta+2))^2}{12})^{n}$

~\\ $L(y;\sigma^{2}) = n(\frac{(2\theta+2))^2}{12})$

~\\ Segun lo anterior podemos concluir que \textbf{{$n(\frac{(2\theta+2))^2}{12})$}} Maximizaria la funcion de L.  

~\\ $Var(Y) = \frac{(2\theta+1-(-1))^2}{12}$
~\\ $Var(Y) = 4\frac{(\theta+2)^2}{12}$
~\\ $Var(Y) = \frac{(\theta+2)^2}{3}$

~\\ Sin embargo no es posible medir sus varianzas puesto que $\theta$ es la $Y_{n}$ muestra de $[Y_{1}, Y_{2}, Y_{3},...,Y_{n}]$ por lo cual. 

~\\ $Var(Y) = \frac{(Y_{(n)}+2)^2}{3}$

\subsection{Punto b.}
~\ Un estimador por el metodo de los momentos para $\theta$ esta dado por:

~\\ $E[Y]= E[\frac{(a+b)}{2}]$
~\\ $E[Y]= E[\frac{(-1+(2\theta+1))}{2}]$
~\\ $E[Y]= E[\frac{(2\theta)}{2}]$

~\\ $E[Y]= E[\theta]$

~\\ $\theta = \frac{\sum(Y_{i})}{n}$ 
~\\ $\hat{\theta} = \hat{Y}$

\bibliography{Bibliografia}
\end{document}